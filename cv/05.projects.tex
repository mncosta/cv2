%-------------------------------------------------------------------------------
%	SECTION TITLE
%-------------------------------------------------------------------------------
\cvsection{Selected Projects}


%-------------------------------------------------------------------------------
%	CONTENT
%-------------------------------------------------------------------------------
\begin{cventries}

  \cventry
	{\href{http://users.isr.ist.utl.pt/~manuel/smartbike/}{\underline{users.isr.ist.utl.pt/~manuel/smartbike/}}} % Job title
	{Smartbike} % Organization
	{Mar 2016 – Sep 2018} % Location
	{} % Date(s)
	{
		\begin{cvitems} % Description(s) of tasks/responsibilities
		Development of different metrics to assess the risk perception of a cyclist when riding a bike in an urban scenario. In this domain, the trajectory of the cyclist is considered, as well as other stationary or moving objects in its vicinity, its change in speed, acceleration and its geographic positioning (given by the smartphone) and the effort/stress of the rider, by analyzing their heart rate variability in an ECG.
		\end{cvitems}
	}
	{
		\#AndroidApp ~
		\#BackendDataPlatform ~
		\#VideoProcessing ~
	}

%-------------------------------------------------------------------------------
  \cventry
	{\href{http://users.isr.ist.utl.pt/~manuel/walkbot/}{\underline{users.isr.ist.utl.pt/~manuel/walkbot/}}} % Job title
	{WalkBot} % Organization
	{May 2017 – Dec 2018} % Location
	{} % Date(s)
	{
		\begin{cvitems} % Description(s) of tasks/responsibilities
			Development of a device capable of automatically digitalize and map a network of sidewalks in a city. This mapping allows for the  development of support applications for guiding pedestrians in cities with various mobility needs. This project was funded by Thales TecInnov 1st Edition.
		\end{cvitems}
	}
	{
		\#StereoVideoProcessing ~
		\#SidewalkMetrics
		\#SidewalkAccessibility
	}

%-------------------------------------------------------------------------------
\cventry
	{\href{ushift.tecnico.ulisboa.pt/bikerider/}{\underline{ushift.tecnico.ulisboa.pt/bikerider/}}} % Job title
	{Bike Rider} % Organization
	{Mar 2018 – Oct 2018} % Location
	{} % Date(s)
	{
		\begin{cvitems} % Description(s) of tasks/responsibilities
			Development of a cyclable network sensorial supported on data captured by cyclists. To create this map we perform cycling maneuvers recognition from inertial data captured by a smartphone.
			This project was funded by Thales TecInnov 2nd Edition.
		\end{cvitems}
	}
	{
		\#CyclingManeuvers ~
		\#ActivityRecognition ~
		\#InertialData
	}

%-------------------------------------------------------------------------------
\cventry
	{Contact Prof. João Paulo Costeira (jpc@isr.tecnico.ulisboa.pt)} % Job title
	{Public Transportation Passenger Tracking \& Counting} % Organization
	{Mar 2017 – Oct 2018} % Location
	{} % Date(s)
	{
		\begin{cvitems} % Description(s) of tasks/responsibilities
			Tracking and counting passengers system developing. Project part of an audit of a public transportation company for a Portuguese transportation national authority.
			3D cameras were used to capture real-time video and automatically track entering or exiting passengers.
			Data from 10 RGBD cameras was stored and edge-processed inside the vehicle.
		\end{cvitems}
	}
	{
		\#TrackingAndCounting ~
		\#3DVideoProcessing ~
	}

%-------------------------------------------------------------------------------
\cventry
{Contact Prof. Filipe Moura (fmoura@tecnico.ulisboa.pt)} % Job title
{Step Home} % Organization
{May 2018 – Dec 2018} % Location
{} % Date(s)
{
	\begin{cvitems} % Description(s) of tasks/responsibilities
		Development of a system capable of guiding people suffering from some sort of neurodegenerative disease to a place they know (a reference point which can be their home, a coffee place they know, a doctor) or to a relative. This project was funded by Thales TecInnov 4th Edition.
	\end{cvitems}
}
{
	\#SmartphoneApp ~
	\#BackendDevelopment ~
}



\end{cventries}