\cvsection{education}

\cvEducation{May 2022 – ONGOING}{DTU Guest PhD Student}{Technical University of Denmark}{Copenhagen, Denmark}
\cvEducationDetails{Spending 8 months working in DTU with Prof. Carlos Lima Azevedo (Machine Learning for Smart Mobility) and Prof. Felix Wilhelm Siebert (Transport Psychology) to further develop my ongoing PhD research and applicability to other contexts.}

\cvEducationNoLocation{JUN 2020 – ONGOING}{EIT Urban Mobility Doctoral Training Network}
\cvEducationDetails{The European Institute of Innovation and Technology Doctoral Training Network (EIT-DTN) seeks to connect PhD candidates with innovation and entrepreneurship in education, research and business. Used as a platform for exchanging inputs and outputs of my research with other practitioners, senior researchers and PhD candidates.}

\cvEducation{OCT 2018 – ONGOING}{MIT Portugal Transportation System Doctoral Program}{Instituto Superior Técnico}{Lisbon, Portugal}
\cvEducationDetails{Research focusing on developing a framework capable of automatically and continuously understand objective and subjective urban cycling safety using inertial, mapping and imagery data. Work supervised by Professor Filipe Moura (IST), Doctor Manuel Marques (IST), and Carlos Lima Azevedo (DTU).}

\cvEducation{SEPT 2011 – MAY 2017}{Integrated master’s in Electrical \& Computer Eng.}{Instituto Superior Técnico}{Lisbon, Portugal}
\cvEducationDetails{Master’s Thesis on: \underline{Video-Based Risk Assessment for Cyclists} supervised by Doctor Manuel Marques and Professor João Paulo Costeira. From videos recorded from a developed Android App, optical flow was computed to discover the Focus of Expansion. This point was used to divide the image into risk levels. Risk situations were classified to then assess other risk events using two metrics. See more under Projects: Smartbike.}

\cvEducation{SEPT 2015 – FEB 2016}{Erasmus+ Exchange Program}{Universitá di Bologna}{Bologna, Italy}
\cvEducationDetails{Spent six months living and studying in Bologna, Italy. Living within a new culture and getting to know about other cultures, allowed me to grow immensely as a person. During this time, I focused on innovation, entrepreneurship, and computer vision using augmented reality.}
