\vspace{7pt}
\cvsection{publications}
% =================================================================== %
%       JOURNALS
% =================================================================== %

\vspace{7pt}\hspace{-5pt}
\begin{tabular*}{1\textwidth}{p{11.cm} p{6.cm}}
        \textcolor{black}{\Large\textbf{Journals}} & \\
    \end{tabular*}
\begin{enumerate}[rightmargin=1.5cm]
\itemsep0em 
\item \textbf{Costa, M.}, Marques, M., Roque, C., Moura, F. (2022). CYCLANDS: Cycling geo-located accidents, their details and severities. Scientific Data, 9(1), 1-9.

\item \textbf{Costa, M.}, Félix, R., Marques, M., Moura, F. (2022). Impact of COVID-19 lockdown on the behavior change of cyclists in Lisbon, using multinomial logit regression analysis. Transportation Research Interdisciplinary Perspectives, 100609.

\item \textbf{Costa, M.}, Marques, M., Moura, F. (2021). A Circuity Temporal Analysis of Urban Street Networks Using Open Data: A Lisbon Case Study. ISPRS International Journal of Geo-Information, 10(7), 453.

\end{enumerate}

% =================================================================== %
%       CONFERENCES WITH PROCEDIA
% =================================================================== %
\vspace{7pt}\hspace{-5pt}
\begin{tabular*}{1\textwidth}{p{11.cm} p{6.cm}}
        \textcolor{black}{\Large\textbf{Conferences with procedia}} & \\
    \end{tabular*}

\begin{enumerate}[rightmargin=1.5cm]
\itemsep0em 
\item Karina Christ, A., \textbf{Costa, M.}, Marques, M., Roque, C., and Moura, F. (2022). Percebendo a segurança objetiva dos ciclistas urbanos: uma revisão sistemática da literatura. 10º Congresso Rodoferroviário Português, Lisboa, July 2022.

\item \textbf{Costa, M.}, Marques, M., Moura, F. (2022). Como é que as redes rodoviárias, pedonais e cicláveis mudam ao longo do tempo? Uma análise da rectilinearidade em Lisboa, Portugal. 10º Congresso Rodoferroviário Português, Lisboa.

\item \textbf{Costa, M.}, Cambra, P., Moura, F., Marques, M. (2019, October). WalkBot: A Portable System to Scan Sidewalks. In 2019 IEEE International Smart Cities Conference (ISC2) (pp. 167-172). IEEE.

\item Cambra, P., \textbf{Costa, M.}, Marques, M. and Moura, F., “WalkBot – Desenvolvimento de um Equipamento de Avaliação da Qualidade da Infra-estrutura Pedonal com Recurso a Processamento de Imagem Tridimensional,” in Congresso Rodoviário Português, Lisbon, Portugal, May 2019

\item \textbf{Costa, M.}, Ferreira, B. Q., Marques, M. (2017, October). A context aware and video-based risk descriptor for cyclists. In 2017 IEEE 20th International Conference on Intelligent Transportation Systems (ITSC) (pp. 1-6). IEEE.

\end{enumerate}



% =================================================================== %
%       MASTERS THESIS
% =================================================================== %
\vspace{7pt}\hspace{-5pt}
\begin{tabular*}{1\textwidth}{p{11.cm} p{6.cm}}
        \textcolor{black}{\Large\textbf{MSc Thesis}} & \\
    \end{tabular*}

\begin{enumerate}[rightmargin=1.5cm]
\itemsep0em 
\item \textbf{Costa, M.}, “Video-Based Risk Assessment for Cyclists,” M.S. thesis, Dept. Elect. And Comp. Eng., UTL, IST, Lisbon, 2017

\end{enumerate}
