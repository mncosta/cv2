\vspace{7pt}
\cvsection{selected projects}

%%%%%%%%%%%%%%%%%%%%%%%%%%%%%%%%%%%%%%%%%%%%%%%%%%%%%%%%%%%%%%%%%%%%%%%%%%%%%%%
\cvProject{MARCH 2016 - September 2018}{Smartbike}
{\textcolor{black}{\href{http://users.isr.ist.utl.pt/~manuel/smartbike/}{\underline{users.isr.ist.utl.pt/~manuel/smartbike/}}} \hspace{5pt}\seticon{faExternalLink}}
{Development of different metrics to assess the risk perception of a cyclist when riding a bike in an urban scenario. In this domain, the trajectory of the cyclist is considered, as well as other stationary or moving objects in its vicinity, its change in speed, acceleration and its geographic positioning (given by the smartphone) and the effort/stress of the rider, by analyzing their heart rate variability in an ECG.}
\cvProjectDetails{Android Application Development}{Development of an Android Application to record several different signals available in the smartphone: Video, Audio, GPS, Acceleration, Angular Velocity and others. The captured data can then be uploaded to a server where the data is stored by user.}
\cvProjectDetails{Backend Data Platform}{Creation of a backend platform that stores all the captured data from the smartphone in a user-based system. This way, some user characteristics are stored (e.g. age, gender, experience on riding a bike), as well as the uploaded timestamped data.}
\cvProjectDetails{Video Processing}{Using the captured and stored data, the video signal is processed to geographically and semantically classify different risk level situations for the cyclist. This processing is performed automatically when a new video is uploaded to the server.}

%%%%%%%%%%%%%%%%%%%%%%%%%%%%%%%%%%%%%%%%%%%%%%%%%%%%%%%%%%%%%%%%%%%%%%%%%%%%%%%
\cvProject{MAY 2017 - December 2018}{WalkBot}
{\textcolor{black}{\href{http://users.isr.ist.utl.pt/~manuel/walkbot/}{\underline{users.isr.ist.utl.pt/~manuel/walkbot/}}} \hspace{5pt}\seticon{faExternalLink}}
{Development of a device capable of automatically digitalize and map a network of sidewalks in a city. This mapping will allow for the further development of support applications with the objective of guiding pedestrians in cities with various mobility needs. This project was funded by Thales TecInnov 1st Edition.}
\cvProjectDetails{Stereo Vision Processing}{Processing a 3D point cloud captured using a stereo vision camera. By recognizing objects, we aim to discover the walkable distance on the sidewalk, the slope of the sidewalk and the size of the sidewalk itself.}

%%%%%%%%%%%%%%%%%%%%%%%%%%%%%%%%%%%%%%%%%%%%%%%%%%%%%%%%%%%%%%%%%%%%%%%%%%%%%%%
\cvProject{MARCH 2018 – OCTOBER 2018}{BikeRider}
{\textcolor{black}{\href{http://ushift.tecnico.ulisboa.pt/bikerider/}{\underline{ushift.tecnico.ulisboa.pt/bikerider/}}} \hspace{5pt}\seticon{faExternalLink}}
{Project to develop a sensorial map of a cyclable network supported on data captured by cyclists. To create this map we perform activity recognition. This project was funded by Thales TecInnov 2nd Edition.}
\cvProjectDetails{Activity Recognition}{Automatic bike maneuvers classification using Deep Learning. A Convolutional Neural Network was used to classify different cycling maneuvers (normal riding, braking, smooth/abrupt turning left/right) based on accelerometer and gyroscope signals captured by a smartphone.}

%%%%%%%%%%%%%%%%%%%%%%%%%%%%%%%%%%%%%%%%%%%%%%%%%%%%%%%%%%%%%%%%%%%%%%%%%%%%%%%
\cvProject{MARCH 2017 – OCTOBER 2018}{Passengers Tracking and Counting}
{Contact Prof. João Paulo Costeira (jpc@isr.ist.utl.pt)}
{Development of a system capable of tracking and counting passengers entering or exiting a vehicle for an audit of a public transportation company for a Portuguese transportation national authority.}
\cvProjectDetails{Data Acquisition System \& Backend System}{3D video acquisition system using 3D cameras and small computer boards. Video was processed in real-time to automatically detect when the vehicle door was opened. Data from 10 RGBD cameras was stored inside the vehicle.}

%%%%%%%%%%%%%%%%%%%%%%%%%%%%%%%%%%%%%%%%%%%%%%%%%%%%%%%%%%%%%%%%%%%%%%%%%%%%%%%
\cvProject{MAY 2018 - December 2018}{StepHome}
{\textcolor{black}{\href{}{Site to be published.}} \hspace{5pt}} %\seticon{faExternalLink}
{Development of a system capable of guiding people suffering from some sort of neurodegenerative disease to a place they know (a reference point which can be their home, a coffee place they know, a doctor) or to a relative. This project was funded by Thales TecInnov 4th Edition.}
\cvProjectDetails{Minimum Viable Product: App \& Backend}{I helped design the minimum valuable product which consisted on a smartphone application and the backend to respond to the application requests.}
